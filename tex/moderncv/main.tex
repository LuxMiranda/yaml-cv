\documentclass[12pt,a4paper,sans]{moderncv} 
\usepackage[document]{ragged2e}

\usepackage{./utfsym}

\moderncvstyle{classic}
\moderncvcolor{green} 

\usepackage{lipsum} 

\usepackage[scale=0.83,top=1.5cm,bottom=1.5cm]{geometry}

%\setlength{\hintscolumnwidth}{3cm} % Changes the width of the dates column

\renewcommand{\baselinestretch}{1.3}

\usepackage{sectsty}
\usepackage{fontspec}

\setsansfont[
BoldFont=EBGaramond-Medium.ttf,
ItalicFont=EBGaramond-Italic.ttf,
BoldItalicFont=EBGaramond-MediumItalic.ttf
]{EBGaramond-Regular.ttf}

\allsectionsfont{\normalsize}

\firstname{{\setsansfont{Forum-Regular.ttf} Lux}}
\familyname{{\setsansfont{Forum-Regular.ttf} Miranda}} 
\pronouns{
  she/they  
}

\title{Lux Miranda Curriculum Vitae}
\extrainfo{
  This abridged CV is \\ 
  current as of \today \\
  To view the full version,\\
visit \href{https://luxmiranda.com/CV}{luxmiranda.com/CV}
\\
Email: \href{mailto:lux.miranda@it.uu.se}{\underline{lux.miranda@it.uu.se}}
 }
\photo[80pt][0.6pt]{lux.png}

\definecolor{LuxOffwhite}{HTML}{F9F9F9}
\pagecolor{LuxOffwhite}


\begin{document}

\makecvtitle

\newcommand{\sitem}{\ \ \ \ $\usym{2727}$ \ }

\section{Education}
\cvitem{2022-2027 \\
 (expected)}{\textbf{PhD in Computer Science}\\
\begin{itemize}\item[$\usym{2727}$]\href{https://en.wikipedia.org/wiki/Uppsala\%5FUniversity}{Uppsala University}, \textit{Sweden, European Union}\\
 \item[$\usym{2727}$]Defense expected May 2027\end{itemize}\vspace*{-\baselineskip}\leavevmode}
\cvitem{2020-2022}{\textbf{Master of Science in Industrial Engineering}\\
\begin{itemize}\item[$\usym{2727}$]\href{https://en.wikipedia.org/wiki/University\%5Fof\%5FCentral\%5FFlorida}{University of Central Florida} (UCF), \textit{Orlando, Florida, USA}\\
 \item[$\usym{2727}$]Honorary \href{https://web.archive.org/web/20220513211358/https://www.cecs.ucf.edu/college-awards-2000th-doctoral-and-10000th-masters-degrees-to-honorary-recipients/}{10,000th master's degree} conferred by the college\\
 \item[$\usym{2727}$]\href{https://stars.library.ucf.edu/etd2020/1054}{Thesis}: \textit{Humans in algorithms, algorithms in humans: Understanding cooperation and creating social AI with causal generative models}\end{itemize}\vspace*{-\baselineskip}\leavevmode}
\cvitem{2016-2020}{\textbf{Bachelor of Science with University Honors, double-major in Computational Mathematics and Computer Science, minor in Anthropology, \textit{Cum Laude}}\\
\begin{itemize}\item[$\usym{2727}$]\href{https://en.wikipedia.org/wiki/Utah\%5FState\%5FUniversity}{Utah State University } (USU) , \textit{Logan, Utah, USA}\\
 \item[$\usym{2727}$]\href{https://doi.org/10.1371/journal.pone.0232609}{Honors thesis}: \textit{Computationally revealing recurrent social formations and their evolutionary trajectories}\end{itemize}\vspace*{-\baselineskip}\leavevmode}
\section{Publications}
\cvitem{2022 \\
 (Accepted)}{Freeman, J., Baggio, J., \textbf{Miranda, L.}, \& Anderies, J.M. (2022). Social infrastructure moderates the energy use of polities. Accepted / in revision, \textit{Cross-Cultural Research}.}
\cvitem{2022 \\
 (Invited; In press)}{\textbf{Miranda, L.}, Garibay O.O., \& Baggio, J. (2022). Evolutionary model discovery of human behavioral factors driving decision-making in an irrigation experiment. Invited and in press for a special issue of the \textit{Journal of Artificial Societies and Social Sciences}.}
\cvitem{2022 \\
 \textit{Invited manuscript}}{\textbf{Miranda, L.} \& Garibay O.O. (2022). Approaching (super)human intent recognition in stag hunt with the Naïve Utility Calculus generative model. \textit{Computational and Mathematical Organization Theory}. \href{https://doi.org/10.1007/s10588-022-09367-y}{https://doi.org/10.1007/s10588-022-09367-y }}
\cvitem{2022}{\textbf{Miranda, L.} (2022). Humans in Algorithms, Algorithms in Humans: Understanding Cooperation and Creating Social AI with Causal Generative Models. \textit{UCF Electronic Theses and Dissertations}. \href{https://stars.library.ucf.edu/etd2020/1054}{https://stars.library.ucf.edu/etd2020/1054 }}
\cvitem{2022}{Bird, D., \textbf{Miranda, L.}, Vander Linden, M. et al. (2022). p3k14c, a synthetic global database of archaeological radiocarbon dates. \textit{Nature Scientific Data}. \href{https://doi.org/10.1038/s41597-022-01118-7}{10.1038/s41597-022-01118-7 }}
\cvitem{2021 \\
 Awarded \textit{Best Human-Autonomy Teaming Paper}}{\textbf{Miranda, L.} \& Garibay O.O. (2021). Multi-agent Naïve Utility Calculus: Intent Recognition in the Stag-Hunt Game. Social, Cultural, and Behavioral Modeling. SBP-BRiMS 2021. Lecture Notes in Computer Science, vol 12720. \href{https://doi.org/10.1007/978-3-030-80387-2\%5F32}{10.1007/978-3-030-80387 232 }}
\cvitem{2020}{\textbf{Miranda, L.} \& Freeman, J. (2020). The two types of society: Computationally revealing recurrent social formations and their evolutionary trajectories. \textit{PLoS One} \href{https://doi.org/10.1371/journal.pone.0232609}{10.1371/journal.pone.0232609 }}

\section{Research Experience}
\cvitem{Summer 2022}{\textbf{PIBBSS Summer Research Fellow}. Awarded the 9,000 USD \href{https://www.pibbss.ai/}{Principles of Intelligent Behavior in Biological and Social Systems } (PIBBSS) summer research fellowship to conduct research on \href{https://en.wikipedia.org/wiki/AI\%5Falignment}{human-aligned AI systems }.}
\cvitem{August 2020 - \\
 May 2022 \\
 (4 semesters)}{\textbf{Graduate Research Assistant.} \textit{University of Central Florida Human-Centered Artificial Intelligence Research Laboratory \& Complex Adaptive Systems Laboratory.} Contributed to the publication of three journal articles, one conference paper, and my master's thesis.}
\cvitem{August 2019 - August 2020 \\
 (1 year)}{\textbf{Undergraduate Research Assistant.} \textit{Utah State University Anthropology Program.} As part of an international archaeological working group known as \href{https://pastglobalchanges.org/science/wg/people-3000/intro}{PEOPLE 3000 }, I helped to create and manage a new \href{https://www.p3k14c.org/}{radiocarbon database } larger and more complete than any other. I also worked to program and test an online social experiment studying cooperation in a common-pool resource management scenario.}
\cvitem{Summer 2019}{\textbf{Peak Summer Research Fellow.} \textit{Utah State University.} One of ten recipients awarded the 4,000 USD \href{https://web.archive.org/web/20220521174643/https://research.usu.edu/peakfellows/index}{Peak Summer Research Fellowship } for highly-engaged undergraduate researchers to conduct work on a proposed project over the summer. The research conducted under this fellowship produced my first publication, listed above.}
\cvitem{Summer 2018}{\textbf{NASA Space Grant Consortium Fellow.} Awarded a 1,600 USD \href{http://www.utahspacegrant.com/about/}{NASA space grant fellowship} to continue work on a CubeSat mission as the software team leader for the USU Get Away Special Microgravity Research team. Managed a team of ten other programmers. Wrote software for a prototype platform that successfully served over a dozen high-altitude balloon flights. The project (\href{https://en.wikipedia.org/wiki/GASPACS}{GASPACS}) was the world's first CubeSat developed entirely by undergraduate students. It successfully served its mission after being launched to the International Space Station as part of the \href{https://en.wikipedia.org/wiki/SpaceX\%5FCRS-24}{SpaceX CRS-24} mission and deployed into low Earth orbit on 26 January 2022.}
\end{document}

